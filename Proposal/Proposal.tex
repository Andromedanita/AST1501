%\documentclass[letterpaper,12pt]{aastex}
\documentclass[preprint]{aastex}
\usepackage[utf8]{inputenc}

\usepackage{rotating}
\usepackage[top=1in, bottom=1in, left=1in, right=1in]{geometry}
\usepackage{graphicx}
\usepackage{setspace}
\usepackage[cdot,mediumqspace,]{SIunits}
\usepackage{hyperref}
\usepackage{mathtools}
\usepackage{url}
\usepackage{authblk}
\usepackage{placeins}
\usepackage{float}
\usepackage{natbib}
\usepackage{multirow}
\usepackage{amsmath}


\title{Dynamics of the Stellar Streams to constrain Milky Way potential}
%\author{Anita Bahmanyar}
\affil{\small {Anita Bahmanyar}}
\affil{\small {Student Number: 998909098}}
\affil{\small {bahmanyar@astro.utoronto.ca}}
\affil{\small {Supervised by: Prof. Jo Bovy}}
\affil{\small {Department of Astronomy and Astrophysics, University of Toronto}}
%\email{bahmanyar@astro.utoronto.ca}
\date{}

\usepackage{graphicx}

%\renewcommand\thesubsection{\alph{subsection}}

\begin{document}


\begin{abstract}
\end{abstract}
\maketitle




\section{Introduction}
Stars in satellite galaxies and clusters get tidally disrupted by their host galaxy as they orbit around it. (ref koposov) The orbit of the tidally disrupted stars is close to that of their progenitor (ref Bowden 2015)
Understanding the physics of these streams help us study the structure of the host galaxy and in particular its potential. Tidal streams can also give us information about the large-scale structure of the Milky Way halo's density distribution and also its small-scale fluctuations. (ref Bovy 2014)

There are a number of streams detected within our Milky Way galaxy which could help us mode the potential of the Milky Way. The most famous example of such case is Saguttarius (Sgr) dwarf galaxy that has been discovered in 1994. The nucleus of the Sgr has survived for many orbits around the Galaxy, while its tidal tails have now been detected over a full $360^{\circ}$ on the sky and provides a strong constraint on the Galaxy�s halo (ref Fellhauer 2006) 
Some of the other detected streams in Milky Way galaxy are GD1 stream, Orphan stream and NGC5466 stream. 
These streams are derived from progenitors with lower mass than that of the Sgr stream so they will be easier to model and study. (ref Bowden 2015)
The difficulty in modelling the streams is that they do not follow a single orbit and this makes it hard to know which fitting of orbits is the best. We can fit more than one single orbit to the streams but it would be computationally expensive which has led to the assumption of fitting one single orbit to the streams. (ref Bovy 2014)
The best kind of streams to help us constrain the potential of Milky Way is a thin one that extends largely on the sky since it allows for accurate modelling if the stream orbit.
GD-1 is a thin stream that is $63^{\circ}$ long on the sky. It is suggested that it is generated from a globular cluster. It is located at  (insert ~) $10 kpc$ from the Sun and (insert ~) $ 15 kpc$ from the Galactic center.(ref Koposov 2010)
There might exist some gaps in the observation data of the tidal streams which could be due to existence of dark matter sub halos, so studying these could help us understand where these dark matter sub halos are and to study the structure of them. But this gap could also be due to the dynamics of the stars in the stream. (ref Bovy 2014)



\section{Current Tools}
I will be using \texttt{galpy} which is a Python package written for galactic dynamics calculations. galpy includes a vast number of functions including different galactic potentials and integrations methods. There are different types of potentials such as one, two and three dimensional potentials which the latter is the more realistic one. 
\\ The units in galpy are not physical units. \texttt{galpy} uses natural units such that circular velocity is one at a cylindrical radius of one and height of zero. (cite galpy paper). So in order to convert to physical units, one needs to multiply the output by the actual values. for instance, one needs to multiply position by $8 \, kpc$ and the velocity by $220 \, kms^{-1}$ in a model where the Sun is assumed to be at $8 \, kpc $ from the Galactic centre and has the circular velocity of $220 \, kms^{-1}$.

\subsection{Coordinate transformations}
We need to convert  from the equatorial coordinates given by declination,$\delta$ and right ascension,$\alpha$ to the cartesian coordinates, x,y,z:

\begin{eqnarray}
x & = & \cos(\delta) \cos(\alpha) \\ \nonumber
y & = & \cos(\delta) \sin(\alpha)  \\ \nonumber
z & = & \sin(\alpha) \\ \nonumber 
\end{eqnarray}

we need to convert the stream coordinates to equatorial coordinate by the transformation below:

\begin{equation}
\begin{bmatrix}
\cos(\phi_1) \cos(\phi_2)
\\ \sin(\phi_1)\cos(\phi_2)
\\ \sin(\phi_2)

\end{bmatrix} = \begin{bmatrix}
-0.4776303088 &  -0.1738432154 & 0.8611897727 \\ 
 0.510844589 & -0.8524449229 & 0.111245042 \\ 
0.7147776536 & 0.493068392 & 0.4959603976 
\end{bmatrix} \times  \begin{bmatrix}
\cos(\alpha) \cos(\delta)
\\ \sin(\alpha) \cos(\delta) 
\\ \sin(\delta)
\end{bmatrix}
\end{equation}

These coordinate transformation are all available in \texttt{galpy} in the utilities folder. (I can use phi12\_to\_lb and returning $\alpha$ and $\delta$ instead of what is being returned now. Then I can convert equatorial to xyz coordinates)

We can convert the Cartesian coordinates to the cylindrical coordinates:

\begin{eqnarray}
R & = & \sqrt{x^2 + y^2} \\ \nonumber
\phi & = & \arctan(\frac{y}{x})  \\ \nonumber
z & = & z, \nonumber
\end{eqnarray}

For initializing the galpy orbit instance we need the initial coordinate of the stream  and its initial velocities in cylindrical coordinates. 
We can also convert the velocities in Cartesian co-ordinates to the velocities in cylindrical coordinate by equation (number it):
\begin{eqnarray}
v_R & = & v_x  \cos(\phi) + v_y  cos(\phi) \\ \nonumber 
v_T & = & -v_x  \cos(\phi) + v_y  cos(\phi) \\ \nonumber
v_z & = &  v_z,  \nonumber
\end{eqnarray}



\section{Method}
\subsection{Likelihood}
We can calculate the likelihood as the probability of getting a y-value at an x-value given a model. In our case, we will consider likelihood as the probability of getting $\phi_2$ at a $\phi_1$ given a model at a given time. In other words, $\mathcal{L} \propto P(\phi_2 \, at \, \phi_1 | \mathrm{model \, at \, time \, t})$.
The likelihood can also be written as:

\begin{eqnarray}
ln \mathcal{L} & = & - \frac{\chi^2}{2} =  \Sigma \frac{(x_{model,i}-x_{data,i})^2}{2\sigma_i^2} \\ \nonumber
\mathcal{L}  & \propto & \int \exp^{\frac{-(\phi_1(t) - {\phi_1^{obs}})^2}{2\sigma_1^2} - \frac{-(\phi_2(t) - {\phi_2^{obs}})^2}{2\sigma_2^2}} dt  \\ \nonumber
\mathcal{L}  & \propto & \sum_i \exp^{\frac{-(\phi_1(t) - {\phi_1^{obs}})^2}{2\sigma_1^2} - \frac{-(\phi_2(t) - {\phi_2^{obs}})^2}{2\sigma_2^2}},  \nonumber
\end{eqnarray }
where $i$ represents each of the data points and $\sigma_i$ is the associated error. We take the sum since we do not have infinite time steps. Also, we need to integrate over time since we do not know the time the data has been taken and we only take an average of it.

\subsection{Brief look at galpy}
We can get the potential from galpy using the code below:

from   galpy  import potential
p = potential.LogarithmicHaloPotential(q=0.9,normalize=1)

where q is the potential flattening and normalizations makes sure that the circular velocity is 1 at R=1 kpc.


The data comes from tables 1-4 in (ref Kpsov 2010). the data is a combination of the Sloan Digital Sky survey (SDSS) and Calar Alto spectroscopy.  (ref Koposov 2010)


\section{Plan}

\section{Timeline}



\end{document}