%\documentclass[letterpaper,12pt]{aastex}
\documentclass[preprint]{aastex}
\usepackage[utf8]{inputenc}
\usepackage{rotating}
\usepackage[top=1in, bottom=1in, left=1in, right=1in]{geometry}
\usepackage{graphicx}
\usepackage{setspace}
\usepackage[cdot,mediumqspace,]{SIunits}
\usepackage{hyperref}
\usepackage{mathtools}
\usepackage{url}
\usepackage{authblk}
\usepackage{placeins}
\usepackage{float}
\usepackage{natbib}
\usepackage{multirow}
\usepackage{amsmath}

\title{Dynamics of the Stellar Streams to constrain Milky Way potential}
\affil{\small {Anita Bahmanyar}}
\affil{\small {Student Number: 998909098}}
\affil{\small {bahmanyar@astro.utoronto.ca}}
\affil{\small {Supervised by: Prof. Jo Bovy}}
\affil{\small {Department of Astronomy and Astrophysics, University of Toronto}}
\date{}

\usepackage{graphicx}

\begin{document}


\begin{abstract}
In this project, we will be looking at stellar streams generated by the tidal disruption of stars in a satellite galaxy or a cluster.
We will fit orbits to the streams using specific potentials to constrain the potential of the Milky Way galaxy. We will use GD-1 data obtained from SDSS and Calar Alto spectroscopy. We will also be looking at the likelihood of the fitted orbit to the GD-1 data points including its position, proper motion and its distance from centre of the Milky Way.

\end{abstract}

\maketitle

\section{Introduction}
Stars in satellite galaxies and clusters get tidally disrupted by their host galaxy as they orbit around it. (ref koposov) The orbit of the tidally disrupted stars is close to that of their progenitor, extending ahead and beyond, making a tail shape. (ref Bowden 2015)
Understanding the physics of these stream orbits help us study the structure of the host galaxy and the shape of galactic halo as well as to constrain its potential. Tidal streams can also give us information about the large-scale structure of the Milky Way halo's density distribution and also its small-scale fluctuations. (ref Bovy 2014)

There are a number of streams detected within our Milky Way galaxy which could help us mode the potential of the Milky Way. The most famous example of such case is Saguttarius (Sgr) dwarf galaxy that has been discovered in 1994. The nucleus of the Sgr has survived for many orbits around the Galaxy, while its tidal tails have now been detected over a full $360^{\circ}$ on the sky and provides a strong constraint on the Galaxy�s halo (ref Fellhauer 2006) 
Some of the other detected streams in Milky Way galaxy are GD1 stream, Orphan stream and NGC5466 stream. 
These streams are derived from progenitors with lower mass than that of the Sgr stream so they will be easier to model and study. (ref Bowden 2015)
The difficulty in modelling the streams is that they do not follow a single orbit and this makes it hard to know which fitting of orbits is the best. We can fit more than one single orbit to the streams but it would be computationally expensive which has led to the assumption of fitting one single orbit to the streams. (ref Bovy 2014)
The best kind of streams to help us constrain the potential of Milky Way is a thin one that extends largely on the sky since it allows for accurate modelling if the stream orbit.

Grillmair Dionatos or GD-1 is a cold thin stream that is $63^{\circ}$ long on the sky. It is suggested that it is generated from a globular cluster but there is no progenitor remnant to confirm this suggestion. It is located at  (insert ~) $10 kpc$ from the Sun and (insert ~) $ 15 kpc$ from the Galactic centre.(ref Koposov 2010)
There might exist some gaps in the observation data of the tidal streams which could be due to existence of dark matter sub halos, so studying these could help us understand where these dark matter sub halos are and to study the structure of them. But this gap could also be due to the dynamics of the stars in the stream. (ref Bovy 2014)

A large number of dark matter sub halos reside within the Milky Way galaxy. Since dark matter does not interact with photos, its not visible to eye and we need indirect ways to confirm its existence. Using streams would help us study these sub halos. Occasionally, there are mysterious gaps in the stellar streams and this might happen when the stream passes behind a dark matter sub halo. 


At a distance of about 8.5 kph, the stream is moving per perpendicular to the line of sight with the velocity of 220 km/s.

A very thin and long stellar stream is the best kind for constraining the galaxy potential since it allows for precise orbital models. (ref koposov 2010). If the stream is very thing, we can make the assumption that the stellar streams are moving along the same orbit even though in general the stars have different energies and angular momentum. (ref koposov 2010)




\section{Current Tools}
I will be using \texttt{galpy} which is a Python package written for galactic dynamics calculations. \texttt{galpy} includes a vast number of functions including different galactic potentials and integrations methods. There are different types of potentials such as one, two and three dimensional potentials which the latter is the more realistic one. 
\\ The units in \texttt{galpy} are not physical units. \texttt{galpy} uses natural units that ensures that the circular velocity is one at a cylindrical radius of one and height of zero. (cite galpy paper). So in order to convert to physical units, one needs to multiply the output by the actual values. for instance, one needs to multiply position by $8 \, kpc$ and the velocity by $220 \, kms^{-1}$ in a model where the Sun is assumed to be at $8 \, kpc $ from the Galactic centre and has the circular velocity of $220 \, kms^{-1}$.




\section{Method}

\subsection{Orbit Fitting}
As mentioned previously, the goal of this project is to constrain the potential of the Milky Way using the GD-1 stream data. To begin, we assume a single-component potential known as flattened logarithmic potential for the Milky Way. This potential is given by equation \ref{eq:logPotential}:

\begin{equation} \label{eq:logPotential}
\Phi(x,y,z) = \frac{V_c^2}{2} \, \mathrm{ln} ({x}^2 + y^2 + (\frac{z}{q_{\Phi}})^2),
\end{equation}
where $V_c$ represents the circular velocity and $q_{\Phi}$ shows the flattening parameter. We can initialize this potential using \texttt{galpy} in the following way:

\begin{verbatim}
from   galpy  import potential
p = potential.LogarithmicHaloPotential(q=0.9,normalize=1)
\end{verbatim}

We then need to initialize the orbit. For initializing the orbit, we need to have initial conditions which are the initial position and velocity components of the stream. The orbit can be initialized in \texttt{galpy} in the following way:

\begin{verbatim}
o = Orbit(vxvv=[R,vR,vT,z,vZ,phi],ro=ro,vo=vo)
\end{verbatim}
where ro and vo are the distance to and the velocity of the stream, respectively. The initial conditions passed to \texttt{galpy} need to be in cylindrical coordinates. However, we can also pass observed quantities such as the right ascension and declination to the function so that it would be:

\begin{verbatim}
o = Orbit(vxvv=[Ra,Dec,distance,pmRa,pmDec,vlos],ro=ro,vo=vo)
\end{verbatim}
where pmRa is the proper motion in right ascension direction ($\mu_{\alpha}$), pmDec is the proper motion in the declination direction ($\mu_{\delta}$) and vlos is the line-of-sight velocity ($V_{los}$).

Once we have the orbit initialized, we can integrate it so get the orbital properties at any given time. These properties include the position, velocity, proper motion, distance, energy and a lot more.

We will be using the GD-1 stellar stream to constrain the potential and the initial conditions of this stream are given in Cartesian coordinates ($x,y,z$), so we need to convert the initial position and velocity components to be in cylindrical coordinates ($R,z,\phi$) to be able to initialize the orbit. This can simply be done by equations \ref{eq:cart_to_cyl} and \ref{eq:vcart_to_cyl} below:

\begin{eqnarray} \label{eq:cart_to_cyl}
R & = & \sqrt{x^2 + y^2} \\ \nonumber
\phi & = & \arctan(\frac{y}{x})  \\ \nonumber
z & = & z, \nonumber
\end{eqnarray}

and 

\begin{eqnarray} \label{eq:vcart_to_cyl}
v_R & = & v_x  \cos(\phi) + v_y  cos(\phi) \\ \nonumber 
v_T & = & -v_x  \cos(\phi) + v_y  cos(\phi) \\ \nonumber
v_z & = &  v_z,  \nonumber
\end{eqnarray}


From orbit integration, we get the proper motion of the stream in Galactic coordinates ($l,b,d$). We also want to have the proper motion of the stream in the stream coordinates ($\mu_{\phi_1},\mu_{\phi_2},V_{los}$). This transformation can be done using equation \ref{eq:proper_lb_phi12}:

\begin{equation}  \label{eq:proper_lb_phi12}
\begin{bmatrix}
\mu_r
\\ \mu_{\phi_1}
\\ \mu_{\phi_2}
\end{bmatrix} = \mathrm{R \,T \, A} \begin{bmatrix}
\mu_r
\\ \mu_{l}
\\ \mu_{b}
\end{bmatrix},
\end{equation}

where the matrices R, T and A are defined below by equations \ref{eq:R}, \ref{eq:T} and \ref{eq:A}, respectively:

\begin{equation} \label{eq:R}
\begin{bmatrix}
  &   &  \\ 
 &  &   \\
  &   &  
\end{bmatrix}
\end{equation}


\begin{equation} \label{eq:T}
\begin{bmatrix}
-0.4776303088  &  0.510844589 & 0.7147776536 \\ 
-0.1738432154 & -0.8524449229 & 0.493068392  \\
 0.8611897727  &  0.111245042 & 0.4959603976 
\end{bmatrix}
\end{equation}

\begin{equation} \label{eq:A}
\begin{bmatrix}
-0.4776303088  &  0.510844589 & 0.7147776536 \\ 
-0.1738432154 & -0.8524449229 & 0.493068392  \\
 0.8611897727  &  0.111245042 & 0.4959603976 
\end{bmatrix}
\end{equation}


\subsection{Likelihood}










\subsection{Coordinate transformations}
In order to initialize stream orbit in \texttt{galpy}, we need to have the initial conditions in cylindrical coordinates. So we will be using a lot of coordinate transformations from stream coordinates to the cylindrical coordinates. One way to do so is to convert the stream coordinates to equaltorial coordinates, $\alpha$ and $\delta$ using equation (?):



\begin{equation}
\begin{bmatrix}
\cos(\alpha) \cos(\delta)
\\ \sin(\alpha) \cos(\delta) 
\\ \sin(\delta)

\end{bmatrix} = \begin{bmatrix}
-0.4776303088  &  0.510844589 & 0.7147776536 \\ 
-0.1738432154 & -0.8524449229 & 0.493068392  \\
 0.8611897727  &  0.111245042 & 0.4959603976 
\end{bmatrix} \times  \begin{bmatrix}
\cos(\phi_1) \cos(\phi_2)
\\\sin(\phi_1)\cos(\phi_2)
\\ \sin(\phi_2),
\end{bmatrix}
\end{equation}

Once we have declination,$\delta$ and right ascension,$\alpha$, we can convert them to Cartesian co-ordinates, xyz by:

\begin{eqnarray}
x & = & \cos(\delta) \cos(\alpha) \\ \nonumber
y & = & \cos(\delta) \sin(\alpha)  \\ \nonumber
z & = & \sin(\alpha) \\ \nonumber 
\end{eqnarray}


\texttt{galpy} also takes the initial positions and velocities in observed coordinates, right ascension and declination and distance to the stream and also the velocities in these coordinates.
We can also initialize orbit such that the returned quantities are in physical units.




\subsection{Likelihood}
We can calculate the likelihood as the probability of getting a y-value at an x-value given a model. In our case, we will consider likelihood as the probability of getting $\phi_2$ at a $\phi_1$ given a model at a given time. In other words, $\mathcal{L} \propto P(\phi_2 \, at \, \phi_1 | \mathrm{model \, at \, time \, t})$.
The likelihood can also be written as:

\begin{eqnarray}
ln \mathcal{L} & = & - \frac{\chi^2}{2} =  \Sigma \frac{(x_{model,i}-x_{data,i})^2}{2\sigma_i^2} \\ \nonumber
\mathcal{L}  & \propto & \int \exp^{\frac{-(\phi_1(t) - {\phi_1^{obs}})^2}{2\sigma_1^2} - \frac{-(\phi_2(t) - {\phi_2^{obs}})^2}{2\sigma_2^2}} dt  \\ \nonumber
\mathcal{L}  & \propto & \sum_i \exp^{\frac{-(\phi_1(t) - {\phi_1^{obs}})^2}{2\sigma_1^2} - \frac{-(\phi_2(t) - {\phi_2^{obs}})^2}{2\sigma_2^2}},  \nonumber
\end{eqnarray }
where $i$ represents each of the data points and $\sigma_i$ is the associated error. We take the sum since we do not have infinite time steps. Also, we need to integrate over time since we do not know the time the data has been taken and we only take an average of it.



where q is the potential flattening and normalizations makes sure that the circular velocity is 1 at R=1 kpc.

We can get a lot of variables out from the initialized orbit including the stream orbit in Galactic and equatorial coordinates, proper motions in the same coordinates, the distance and a lot more.


The data comes from tables 1-4 in (ref Kpsov 2010). The data is a combination of the Sloan Digital Sky survey (SDSS) and Calar Alto spectroscopy.  (ref Koposov 2010)


\section{Plan}


\section{Timeline}



\end{document}

